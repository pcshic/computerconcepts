\ifx\ntpcNinetyThree\undefined[93學年基北區] \fi
\label{ntpc-93-p42} 利用下列的演算法追蹤 (traverse) 右下方的二元樹 (binary tree) T 中的節點 (node) 並列印出來,請問其節點資料 A、B、C、D、E、$+$、$-$、$*$、$/$ 印出的順序為 \underlineblank{\ref{ntpc-93-p42}}。\customlabel{ntpc-93-a42}{$A*B-C+D/E$}
\begin{figure}[h!]
  \begin{center}
  \begin{tikzpicture}
    \SetVertexMath
    \tikzset{VertexStyle/.style = {
      shape = circle,%
      inner sep = 0pt,%
      outer sep = 0pt,%
      minimum size = 28pt,%
      draw}}
    \Vertex[x=0,y=0]{+}
    \Vertex[x=0,y=0,LabelOut,Lpos=10,Ldist=.5cm]{T}
    \Vertex[x=-2,y=-1]{-}
    \Vertex[x=2,y=-1]{/}
    \Vertex[x=-3,y=-2.5]{*}
    \Vertex[x=-1,y=-2.5]{C}
    \Vertex[x=1,y=-2.5]{D}
    \Vertex[x=3,y=-2.5]{E}
    \Vertex[x=-4,y=-4]{A}
    \Vertex[x=-2,y=-4]{B}
    \Edge(+)(-)
    \Edge(+)(/)
    \Edge(-)(*)
    \Edge(-)(C)
    \Edge(/)(D)
    \Edge(/)(E)
    \Edge(*)(A)
    \Edge(*)(B)
  \end{tikzpicture}
  \end{center}
\end{figure}
\algrenewcommand\algorithmicprocedure{\textbf{Procedure}}
\algrenewcommand\algorithmicif{\textbf{If}}
\begin{algorithm}[h!]
  \begin{algorithmic}
  \Procedure{order}{$T$}
    \If{$T\neq{0}$}
      \State \Call{order}{LeftChild($T$)};
      \State print(DATA($T$));
      \State \Call{order}{RightChild($T$)};
    \EndIf
  \EndProcedure
  \end{algorithmic}
\end{algorithm}
上述演算法中,order 為副程式的名字,T 表一個二元樹,每一個 T 中的節點皆包含三個部份,LeftChild、DATA、以及 RightChild。LeftChild 與 RightChild 分別為指向 T 的左子樹 (left subtree) 與右子樹 (right subtree) 的指標 (pointer),而 data 則是存節點的資料,在本題中分別為 A、B、C、D、E、$+$ 、$-$、$*$、$/$。