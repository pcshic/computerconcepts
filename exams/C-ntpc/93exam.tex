\documentclass[12pt,a4paper]{report}

%% === CJK 套件 ===
\usepackage{CJK,CJKnumb}                     % 中文套件
%% === AMS 標準套件 ===
\usepackage{amsmath,amsfonts,amssymb,amsthm} % 數學符號
%% === 章節內容 ===
\usepackage{titletoc,titlesec}               % titletoc 目錄修改套件, titlesec 美化章節標題套件
\usepackage{imakeidx}                        % 索引
\usepackage{enumitem}
%% ===  ===
\usepackage[chapter]{algorithm}              % 演算法套件
\usepackage[noend]{algpseudocode}            % pseudocode 套件
\usepackage{listings}                        % 程式碼
%% === TikZ 套件 ===
\usepackage{tikz,tkz-graph,tkz-berge}        % 繪圖

%\linespread{1.24}

%% === 設定頁面格式 ===
%\hoffset         = 10pt                      % 水平位移,預設為 0pt
\voffset         = -15pt                     % 垂直位移,預設為 0pt
\oddsidemargin   = 0pt                       % 預設為 31pt
%\topmargin       = 20pt                      % 預設為 20pt
%\headheight      = 12pt                      % header 的高度,預設為 12pt
%\headsep         = 25pt                      % header 和 body 的距離,預設為 25pt
\textheight      = 620pt                     % body 內文部分的高度,預設為 592pt
\textwidth       = 450pt                     % body 內文部分的寬度,預設為 390pt
%\marginparsep    = 10pt                      % margin note 和 body 的距離,預設為 10pt
%\marginparwidth  = 35pt                      % margin note 的寬度,預設為 35pt
%\footskip        = 30pt                      % footer 高度 + footer 和 body 的距離,預設為 30pt

\newlist{optionlist}{enumerate}{1}
\setlist[optionlist]{label=(\Alph*),before=\raggedright}
\makeatletter
\newcommand{\customlabel}[2]{%
\protected@write \@auxout {}{\string \newlabel {#1}{{#2}{}}}}
\makeatother
\newcommand{\underlineblank}[1]{\underline{\hspace{0.4cm}{({#1})}\hspace{0.4cm}}}

\begin{document}
\begin{CJK}{UTF8}{bkai}

%% === 常用的指令,替換成中文 ===
\renewcommand{\figurename}{圖}
\renewcommand{\tablename}{表}
\renewcommand{\contentsname}{目~錄~}
\renewcommand{\listfigurename}{插~圖~目~錄}
\renewcommand{\listtablename}{表~格~目~錄}
\renewcommand{\appendixname}{附~錄}
%\renewcommand{\refname}{參~考~資~料}    % article
\renewcommand{\bibname}{參~考~文~獻}     % book
\renewcommand{\indexname}{索~引}
\renewcommand{\today}{\number\year~年~\number\month~月~\number\day~日}
%\newcommand{\zhtoday}{\CJKdigits{\the\year}年\CJKnumber{\the\month}月\CJKnumber{\the\day}日}

\title{九十三學年度台灣省高中資訊學科能力競賽}
\date{}
\maketitle

\begin{enumerate}
\item 關於映像管螢幕的敘述,下列那一項是錯的?
  \begin{optionlist}
  \item 映像管螢幕中的電子槍所射出的三個電子束控制的分別是紅、綠、藍三個顏色。
  \item 所謂 17 吋螢幕中的「17 吋」指的是映像管對角線的長度。
  \item 螢幕的水平掃描頻率指的是螢幕的更新率,也就是螢幕每秒鐘可以更新幾張畫面,單位為赫茲 (Hz)。\label{north-93-a1}
  \item 解析度是螢幕能夠顯示的像素 (pixel) 數目,解析度愈高,螢幕上顯示的圖形就會愈細緻。
  \end{optionlist}
\item 關於印表機的敘述,下列那一項是錯的?
  \begin{optionlist}
  \item 印表機可以分為撞擊式 (impact) 與非撞擊式 (nonimpact) 二種。
  \item 噴墨印表機屬於撞擊式印表機。\label{north-93-a2}
  \item 印表機的列印速度是以 ppm (page per minute) 來描述,意指每分鐘可以列印幾頁。 
  \item 印表機的列印品質是以 dpi (dot per inch) 來描述,意指在一英吋的寬度內,可以細分成幾個點。
  \end{optionlist}
\item 下列何者敘述不正確?
  \begin{optionlist}
  \item SRAM 不需做更新 (reflesh) 的動作,DRAM 則需要。
  \item SRAM 在停電後儲存的資料還在,DRAM 在停電後儲存的資料就不見了。\label{north-93-a3}
  \item SRAM 和 DRAM 都可以當做電腦的記憶體。
  \item ROM 在電腦關機後儲存的資料不會消失。
  \end{optionlist}
\item 電腦中的中央處理單位 CPU 主要由下列那兩個單位所組成?
  \begin{optionlist}
  \item 輸入單元與輸出單元
  \item 記憶單元與控制單元
  \item 算術邏輯單元與輸出單元
  \item 控制單元與算術邏輯單元\label{north-93-a4}
  \end{optionlist}
\item 一般所謂 32 位元或 64 位元微處理機 (Microprocessor) 是基於下列何者而稱呼的?
  \begin{optionlist}
  \item 暫存器 (Register) 數目
  \item 位址匯流排 (Address Bus)
  \item 控制匯流排 (Control Bus)
  \item 資料匯流排 (Data Bus)\label{north-93-a5}
  \end{optionlist}
\item 那一種錯誤在程式編譯過程中會被發現?
  \begin{optionlist}
  \item 邏輯錯誤
  \item 資料錯誤
  \item 使用者輸入錯誤
  \item 語法錯誤\label{north-93-a6}
  \end{optionlist}
\item 若有 15 筆已排序好的整數資料,欲從中搜尋某一整數是否存在該數列中,下列何者敘述不正確?
  \begin{optionlist}
  \item 以二元搜尋法 (binary search),最多可能需做 5 次整數比較。\label{north-93-a7}
  \item 以二元搜尋法 (binary search),最少要做 1 次整數比較。
  \item 以循序搜尋法 (sequential search),最少要做 1 次整數比較。
  \item 以循序搜尋法 (sequential search),最多可能需做 15 次整數比較。
  \end{optionlist}
\item 關於陣列 (array) 的描述,下列何者是錯的?
  \begin{optionlist}
  \item 同一陣列常被用來存放多個不同型態的資料。\label{north-93-a8}
  \item 一個陣列只有一個名稱,但可以透過索引 (index) 來存取陣列元素。
  \item 陣列的宣告會佔用連續的記憶體空間。
  \item 陣列可以不只一維 (1D),必要時也可以宣告二維或三維陣列,甚至是多維陣列。
  \end{optionlist}
\item 大明打字速度平均一分鐘 60 個中文字,如果要打完一份足夠將一片磁片 (1.44MB) 填滿的文件,約需耗時多久? (選最近值)
  \begin{optionlist}
  \item 250 小時\label{north-93-a9}
  \item 300 小時
  \item 25 小時
  \item 50 小時
  \end{optionlist}
\item 如果將下列二進位數字 ${(101101)}_{2}-{(100010)}_{2}$ 以 2 的補數加法來計算則可得:
  \begin{optionlist}
  \item 1001111
  \item 001011\label{north-93-a10}
  \item 101111
  \item 001111
  \end{optionlist}
\item 電腦記憶體在儲存數值資料時通常會以有限的位元來表示,故數值資料在儲存時會受到限制。當要儲存數值太大時,就會有何種錯誤現象?
  \begin{optionlist}
  \item 溢位 (Overflow)\label{north-93-a11}
  \item 虧位 (Underfiow)
  \item 捨入誤差 (Round-off)
  \item 捨去誤差 (Truncation)
  \end{optionlist}
\item 下列何種資料結構的資料存取方式是先進先出?
  \begin{optionlist}
  \item 堆疊 (Stack)
  \item 佇列 (Queue)\label{north-93-a12}
  \item 陣列 (Array)
  \item 以上皆非
  \end{optionlist}
\item 將字串 ABACABCD 依序由左至右全部輸入堆疊 (stack) 後再輸出,其輸出結果為何?
  \begin{optionlist}
  \item ABACABCD
  \item AAABBCCD
  \item DCBACABA\label{north-93-a13}
  \item ABCDABAC
  \end{optionlist}
\item 下列何者不相等?
  \begin{optionlist}
  \item ${(79)}_{10}$
  \item ${(4F)}_{16}$
  \item ${(127)}_{8}$\label{north-93-a14}
  \item ${(1001111)}_{2}$
  \end{optionlist}
\item 若一部電腦有 28 條位址匯流排,32 條資料匯流排,在主記憶體中每一個位元組 (byte) 都需要一個位址編碼的前提下,該電腦的主記憶體最多可擴充到多少位元組 (byte)?
  \begin{optionlist}
  \item 256 Mbyte\label{north-93-a15}
  \item 256 Kbyte
  \item 8 Mbyte
  \item 8 Gbyte
  \end{optionlist}
\item 下列那一組二進位數字在做完 OR 的運算後其結果與其他組的不同?
  \begin{optionlist}
  \item 01010101 與 10101010	
  \item 11011101 與 10101010
  \item 01010101 與 00100010\label{north-93-a16}
  \item 11011101 與 00100010
  \end{optionlist}
\item 若用 8 bits 來儲存使用二的補數表示法表示的二進位數字,請問它能儲存何種範圍的數字?
  \begin{optionlist}
  \item 0 至 +256
  \item -128 至 +128
  \item -127 至 +128
  \item -128 至 +127\label{north-93-a17}
  \end{optionlist}
\item 微軟公司 Windows XP 作業系統的 XP 是什麼的縮寫?
  \begin{optionlist}
  \item eXPert
  \item eXPerimental
  \item eXPensive
  \item eXPerience\label{north-93-a18}
  \end{optionlist}
\item 下列何者不是壓縮檔的副檔名?
  \begin{optionlist}
  \item BMP\label{north-93-a19}
  \item ZIP
  \item RAR
  \item GZ
  \end{optionlist}
\item 下列何者為物件導向程式語言?
  \begin{optionlist}
  \item ADA
  \item PROLOG
  \item C++\label{north-93-a20}
  \item PASCAL
  \end{optionlist}
\item 已知下列三個敘述皆成立:\\
"張三喜歡吃西瓜或李四喜歡吃西瓜",\\
"若王五不喜歡吃西瓜則李四不喜歡吃西瓜",\\
"若張三喜歡吃西瓜則李四喜歡吃西瓜"。\\
下列何者敘述一定會成立?
  \begin{optionlist}
  \item 張三喜歡吃西瓜
  \item 李四不喜歡吃西瓜
  \item 李四喜歡吃西瓜\label{north-93-a21}
  \item 王五不喜歡吃西瓜
  \end{optionlist}
\item 關於 XML (extensible markup language),下列何者敘述正確?
  \begin{optionlist}
  \item 是一種針對工程計算用的程式語言
  \item 是一種網頁語言\label{north-93-a22}
  \item 是一種擴充性的繪圖程式語言
  \item 是一種擴充性的人工智慧程式語言
  \end{optionlist}
\item 下列何者可以把例如 www.yahoo.com 這樣的網站名稱轉為 IP 位址表示?
  \begin{optionlist}
  \item proxy 伺服器
  \item Email 伺服器
  \item DNS 伺服器\label{north-93-a23}
  \item WWW 伺服器
  \end{optionlist}
\item 下列何者不是合理的 IP Address?(以 IPv4 來判斷)
  \begin{optionlist}
  \item 140.122.76.2
  \item 140.122.76.128
  \item 140.122.76.247
  \item 140.122.76.256\label{north-93-a24}
  \end{optionlist}
\item 要在 Internet 上利用全球資訊網 (WWW) 查詢某一特定文件時,最快的方式是直接輸入該文件的網址 (URL),例如:http://www.ice.ntnu.edu.tw:100/ICE/index.html。請問上述的範例中,定義通訊協定的是那一部份?
  \begin{optionlist}
  \item http\label{north-93-a25}
  \item www.ice.ntnu.edu.tw
  \item ICE
  \item index.html
  \end{optionlist}
\item TCP/IP 是一種
  \begin{optionlist}
  \item 程式語言
  \item 作業系統
  \item 網路設備
  \item 通訊協定\label{north-93-a26}
  \end{optionlist}
\item 下列何者不是副程式的特性?
  \begin{optionlist}
  \item 增進程式的結構化及模組化
  \item 提高程式的可讀性
  \item 程式的除錯不易\label{north-93-a27}
  \item 需要花較多執行時間
  \end{optionlist}
\item 下列何者不是作業系統?
  \begin{optionlist}
  \item UNIX
  \item OS/2
  \item WindowsXP
  \item OfficeXP\label{north-93-a28}
  \end{optionlist}
\item 下列哪一項不屬於作業系統的工作範圍?
  \begin{optionlist}
  \item 輸出入管理
  \item 編譯程式\label{north-93-a29}
  \item 記憶體管理
  \item 分配電腦資源
  \end{optionlist}
\item \label{north-93-p30} ${(10101101)}_{2}$ 的十進位表示法為 \underlineblank{\ref{north-93-p30}}。\customlabel{north-93-a30}{173}
\item \label{north-93-p31} 將 -128 以八位元的二補數 (two's complement) 表示的結果為何? \underlineblank{\ref{north-93-p31}}\customlabel{north-93-a31}{10000000}
\item \label{north-93-p32} 在十六進位中 ${(43)}_{16}$ 與 ${(39)}_{16}$ 相加後,其結果值為 \underlineblank{\ref{north-93-p32}}。\customlabel{north-93-a32}{${(7C)}_{16}={(124)}_{10}$}
\item \label{north-93-p33} ${(110)}_{x}={(70)}_{8}$,請問 x 之值應為何? \underlineblank{\ref{north-93-p33}}\customlabel{north-93-a33}{7}
\item \label{north-93-p34} 若將運算式 $4*8^5+8^3+2*8+1$ 的運算結果以二進位數表示,則該二進位數中共有 \underlineblank{\ref{north-93-p34}} 個 ``1"。\customlabel{north-93-a34}{4}
\item \label{north-93-p35} 考慮下列的遞迴函式:$F(0)=0,F(1)=1$,對於 $n\geq{2}$, $F(n)=F(n-1)+F(n-2)$。請問 $F(8)=$\underlineblank{\ref{north-93-p35}}。\customlabel{north-93-a35}{21}
\item \label{north-93-p36} 如果以泡沫排序法 (Bubble Sort) 來將數串 (30, 17, 8, 24, 53, 10, 15, 28) 由大排到小,則共要 \underlineblank{\ref{north-93-p36}} 次數字對調 (Swap) 方可完成?\customlabel{north-93-a36}{14}
\item \label{north-93-p37} 給定 4096 筆數字資料存在一個陣列中,先用泡沫排序法 (Bubble Sort) 由小至大排序,再用二分法搜尋 ``1498" 是否存在於陣列中,則最多需比較 \underlineblank{\ref{north-93-p37}} 次就可判定 ``1498" 是否存在?\customlabel{north-93-a37}{13}
\item \label{north-93-p38} 如果要利用電腦播放影片,可將影片檔案存於光碟片中。因為影片檔案本身十分龐大,為了便於儲存,我們會將檔案進行壓縮處理。請問目前最常用的影片資料壓縮格式是什麼? \underlineblank{\ref{north-93-p38}}\customlabel{north-93-a38}{MPEG, AVI}
\item \label{north-93-p39} 電腦中所謂全彩影像是以 \underlineblank{\ref{north-93-p39}} 個位元組 (byte) 來表示影像中每個像素的顏色。\customlabel{north-93-a39}{3}
\item \label{north-93-p40} 若以 9600 bps 的傳輸速率傳送 6000 個 Big-5 碼中文字,須多少秒的傳輸時間? \underlineblank{\ref{north-93-p40}} 秒。\customlabel{north-93-a40}{10}
\item \label{north-93-p41} 在樹狀結構 (tree) 中,假設每個節點 (node) 都有 k 個子節點的連結 (link),我們稱這種樹為 k-ary tree。假設 k-ary tre 中每一個節點都可存放一筆資料。若需利用此種樹存放 $k^5$ 筆資料,則從根 (root) 節點算起為第一層,根節點的子節點則為第二層,以此類推,此樹最少需建至第 \underlineblank{\ref{north-93-p41}} 層 (level) 才能存放所有的資料。\customlabel{north-93-a41}{6}
\item \label{north-93-p42} 利用下列的演算法追蹤 (traverse) 右下方的二元樹 (binary tree) T 中的節點 (node) 並列印出來,請問其節點資料 A、B、C、D、E、$+$、$-$、$*$、$/$ 印出的順序為 \underlineblank{\ref{north-93-p42}}。\customlabel{north-93-a42}{$A*B-C+D/E$}
\begin{figure}[h!]
  \begin{center}
  \begin{tikzpicture}
    \SetVertexMath
    \tikzset{VertexStyle/.style = {
      shape = circle,%
      inner sep = 0pt,%
      outer sep = 0pt,%
      minimum size = 28pt,%
      draw}}
    \Vertex[x=0,y=0]{+}
    \Vertex[x=0,y=0,LabelOut,Lpos=10,Ldist=.5cm]{T}
    \Vertex[x=-2,y=-1]{-}
    \Vertex[x=2,y=-1]{/}
    \Vertex[x=-3,y=-2.5]{*}
    \Vertex[x=-1,y=-2.5]{C}
    \Vertex[x=1,y=-2.5]{D}
    \Vertex[x=3,y=-2.5]{E}
    \Vertex[x=-4,y=-4]{A}
    \Vertex[x=-2,y=-4]{B}
    \Edge(+)(-)
    \Edge(+)(/)
    \Edge(-)(*)
    \Edge(-)(C)
    \Edge(/)(D)
    \Edge(/)(E)
    \Edge(*)(A)
    \Edge(*)(B)
  \end{tikzpicture}
  \end{center}
\end{figure}
\algrenewcommand\algorithmicprocedure{\textbf{Procedure}}
\begin{algorithmic}[h]
  \begin{algorithm}
  \Procedure{order}{$T$}
    \If{$T\neq{0}$}
      \State \Call{order}{LeftChild($T$)};
      \State print(DATA($T$));
      \State \Call{order}{RightChild($T$)};
    \EndIf
  \EndProcedure
  \end{algorithm}
\end{algorithmic}
上述演算法中,order 為副程式的名字,T 表一個二元樹,每一個 T 中的節點皆包含三個部份,LeftChild、DATA、以及 RightChild。LeftChild 與 RightChild 分別為指向 T 的左子樹 (left subtree) 與右子樹 (right subtree) 的指標 (pointer),而 data 則是存節點的資料,在本題中分別為 A、B、C、D、E、$+$、$-$、$*$、$/$。
\item \label{north-93-p43} 承上題,若將演算法改為
\newpage
\begin{algorithmic}[h]
  \begin{algorithm}
  \Procedure{order}{$T$}
    \If{$T\neq{0}$}
      \State \Call{order}{RightChild($T$)};
      \State print(DATA($T$));
      \State \Call{order}{LeftChild($T$)};
    \EndIf
  \EndProcedure
  \end{algorithm}
\end{algorithmic}
請問其節點資料印出的順序又為何? \underlineblank{\ref{north-93-p43}}。\customlabel{north-93-a43}{$E/D+C-B*A$}
\end{enumerate}

\newpage

\section*{參考答案}

\begin{table}[h]
  \center
  \begin{tabular}{|c|c|c|c|c|c|c|c|c|c|}
  \hline
  題號 & 答案 & 題號 & 答案 & 題號 & 答案 & 題號 & 答案 & 題號 & 答案\\
  \hline\hline
  \textbf{1}  & \ref{north-93-a1} & \textbf{2}  & \ref{north-93-a2} & \textbf{3}  & \ref{north-93-a3} & \textbf{4}  & \ref{north-93-a4} & \textbf{5}  & \ref{north-93-a5}\\
  \hline
  \textbf{6}  & \ref{north-93-a6} & \textbf{7}  & \ref{north-93-a7} & \textbf{8}  & \ref{north-93-a8} & \textbf{9}  & \ref{north-93-a9} & \textbf{10} & \ref{north-93-a10}\\
  \hline
  \textbf{11} & \ref{north-93-a11} & \textbf{12} & \ref{north-93-a12} & \textbf{13} & \ref{north-93-a13} & \textbf{14} & \ref{north-93-a14} & \textbf{15} & \ref{north-93-a15}\\
  \hline
  \textbf{16} & \ref{north-93-a16} & \textbf{17} & \ref{north-93-a17} & \textbf{18} & \ref{north-93-a18} & \textbf{19} & \ref{north-93-a19} & \textbf{20} & \ref{north-93-a20}\\
  \hline
  \textbf{21} & \ref{north-93-a21} & \textbf{22} & \ref{north-93-a22} & \textbf{23} & \ref{north-93-a23} & \textbf{24} & \ref{north-93-a24} & \textbf{25} & \ref{north-93-a25}\\
  \hline
  \textbf{26} & \ref{north-93-a26} & \textbf{27} & \ref{north-93-a27} & \textbf{28} & \ref{north-93-a28} & \textbf{29} & \ref{north-93-a29} & &\\
  \hline
  \end{tabular}
\end{table}

\begin{table}[h]
  \center
  \begin{tabular}{|c|c|c|c|}
  \hline
  題號 & 答案 & 題號 & 答案\\
  \hline\hline
  \textbf{30} & \ref{north-93-a30} & \textbf{31} & \ref{north-93-a31}\\
  \hline
  \textbf{32} & \ref{north-93-a32} & \textbf{33} & \ref{north-93-a33}\\
  \hline
  \textbf{34} & \ref{north-93-a34} & \textbf{35} & \ref{north-93-a35}\\
  \hline
  \textbf{36} & \ref{north-93-a36} & \textbf{37} & \ref{north-93-a37}\\
  \hline
  \textbf{38} & \ref{north-93-a38} & \textbf{39} & \ref{north-93-a39}\\
  \hline
  \textbf{40} & \ref{north-93-a40} & \textbf{41} & \ref{north-93-a41}\\
  \hline
  \textbf{42} & \ref{north-93-a42} & \textbf{43} & \ref{north-93-a43}\\
  \hline
  \end{tabular}
\end{table}

\end{CJK}
\end{document}