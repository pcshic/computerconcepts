\documentclass[12pt,a4paper,oneside]{book}
\def \allfiles {}

%% === CJK 套件 ===
\usepackage{CJK,CJKnumb}                     % 中文套件
%% === AMS 標準套件 ===
\usepackage{amsmath,amsfonts,amssymb,amsthm} % 數學符號
%% === 章節內容 ===
\usepackage{titletoc,titlesec}               % titletoc 目錄修改套件, titlesec 美化章節標題套件
\usepackage{imakeidx}                        % 索引
%% ===  ===
\usepackage[chapter]{algorithm}              % 演算法套件
\usepackage[noend]{algpseudocode}            % pseudocode 套件
\usepackage{listings}                        % 程式碼
%% === TikZ 套件 ===
\usepackage{tikz,tkz-graph,tkz-berge}        % 繪圖

\linespread{1.24}

%% === 設定頁面格式 ===
%\hoffset         = 10pt                      % 水平位移,預設為 0pt
\voffset         = -15pt                     % 垂直位移,預設為 0pt
\oddsidemargin   = 0pt                       % 預設為 31pt
%\topmargin       = 20pt                      % 預設為 20pt
%\headheight      = 12pt                      % header 的高度,預設為 12pt
%\headsep         = 25pt                      % header 和 body 的距離,預設為 25pt
\textheight      = 620pt                     % body 內文部分的高度,預設為 592pt
\textwidth       = 450pt                     % body 內文部分的寬度,預設為 390pt
%\marginparsep    = 10pt                      % margin note 和 body 的距離,預設為 10pt
%\marginparwidth  = 35pt                      % margin note 的寬度,預設為 35pt
%\footskip        = 30pt                      % footer 高度 + footer 和 body 的距離,預設為 30pt

\makeindex[name=noun]        % 索引生成

\begin{document}
\begin{CJK}{UTF8}{bkai}

%% === 常用的指令,替換成中文 ===
\renewcommand{\figurename}{圖}
\renewcommand{\tablename}{表}
\renewcommand{\contentsname}{目~錄~}
\renewcommand{\listfigurename}{插~圖~目~錄}
\renewcommand{\listtablename}{表~格~目~錄}
\renewcommand{\appendixname}{附~錄}
%\renewcommand{\refname}{參~考~資~料}    % article
\renewcommand{\bibname}{參~考~文~獻}     % book
\renewcommand{\indexname}{索~引}
\renewcommand{\today}{\number\year~年~\number\month~月~\number\day~日}
%\newcommand{\zhtoday}{\CJKdigits{\the\year}年\CJKnumber{\the\month}月\CJKnumber{\the\day}日}

%% === 
\floatname{algorithm}{演算法}

%% ===
\setcounter{secnumdepth}{3}                                 % 設定計數到 subsubsection
\renewcommand{\thepart}{第\CJKnumber{\arabic{part}}部分}
\renewcommand{\thechapter}{\arabic{chapter}}
\renewcommand{\thesection}{\arabic{section}}                % 改 section 為 1, 2, 3 非 1.1, 1.2, 1.3
\renewcommand{\thesubsection}{\arabic{subsection}}          % subsection 也改一改
\renewcommand{\thesubsubsection}{\arabic{subsubsection}}    % subsubsection 也改一改

%% === 設定章節標題 (配合 titlesec) ===
\titleformat{\part}[display]
	{\center\huge\bfseries}
	{\thepart}
	{1em}
	{\Huge}
%
\titleformat{\chapter}[display]
	{\bf\Large}
	{第~\CJKnumber{\thechapter}~章}
	{1ex}
	{\Huge}
	[\vspace{2ex}]
%
\titleformat{\section}{\Large\bfseries}{第\CJKnumber{\thesection}節}{1em}{}
%
\titleformat{\subsection}{\large\bfseries}{\CJKnumber{\thesubsection}、}{0.5em}{}
%
\titleformat{\subsubsection}{\bfseries}{(\CJKnumber{\thesubsubsection})}{0.5em}{}
% \titlespacing{\subsubsection}{0pt}{5pt}{-10pt}

%% === 設定目錄標題 (配合 titletoc) ===
\titlecontents{part}[0em]
{\center\Large}
{}
{}
{}
%
\titlecontents{chapter}[0em]
{}
{\large\bf{第\CJKnumber{\thecontentslabel}章~~}}
{}{~~\titlerule*{.}\bf\contentspage}
%
\titlecontents{section}[4em]
{}
{第\CJKnumber{\thecontentslabel}節\quad}
{}{~~\titlerule*{.} \contentspage}
%
\titlecontents{subsection}[8em]
{}
{\CJKnumber{\thecontentslabel}、}
{}{~~\titlerule*{.} \contentspage}

%% === 定義 ===
\newtheorem{myrule}{\begin{CJK}{UTF8}{bkai}原理\end{CJK}}[section]
\newtheorem{mythm}{\begin{CJK}{UTF8}{bkai}定理\end{CJK}}[section]
\newtheorem{mydef}{\begin{CJK}{UTF8}{bkai}定義\end{CJK}}[section]
\newtheorem*{mydef*}{\begin{CJK}{UTF8}{bkai}定義\end{CJK}}
\newtheorem{mypropo}{\begin{CJK}{UTF8}{bkai}性質\end{CJK}}[section]
\newtheorem{myquest}{\begin{CJK}{UTF8}{bkai}例題\end{CJK}}[section]
\newtheorem{myexe}{\begin{CJK}{UTF8}{bkai}練習題\end{CJK}}[subsection]
\numberwithin{equation}{section}
\renewenvironment{proof}{\begin{CJK}{UTF8}{bkai}\textbf{證明}\end{CJK}}{\qed}
\newenvironment{mysol}{\begin{CJK}{UTF8}{bkai}\textbf{解答}\end{CJK}}{\qed}

\title{Computer Concepts\\計算機概論}
\author{許胖}
\maketitle
\tableofcontents

\chapter{導論}

\section{基礎知識}

\subsection{單位前綴}

\paragraph{}\textbf{單位前綴}:幾乎所有的單位都是由「\textbf{前綴}」和「\textbf{單位}」本身所組成的,例如:毫米 (mm, millimeter)、千瓦 (kw, kilowatt)、奈秒 (ns, nanosecond)、...等等。下表列出常用的單位前綴:

\begin{table}[h!]
\centering
\label{tbl:unit-prefix}
\begin{tabular}{|llll|llll|}
\hline
前綴    & 英文縮寫 & 中文縮寫 & 數值   & 前綴    & 英文縮寫 & 中文縮寫 & 數值    \\
\hline
yotta & Y    & 佑    & $10^{24}$ & yocto & y    & 攸    & $10^{-24}$\\
zetta & Z    & 皆    & $10^{21}$ & zepto & z    & 介    & $10^{-21}$\\
exa   & E    & 艾    & $10^{18}$ & atto  & a    & 阿    & $10^{-18}$\\
peta  & P    & 拍    & $10^{15}$ & femto & f    & 飛    & $10^{-15}$\\
tera  & T    & 兆    & $10^{12}$ & pico  & p    & 皮    & $10^{-12}$\\
giga  & G    & 吉    & $10^{9}$  & nano  & n    & 奈    & $10^{-9}$\\
mega  & M    & 百萬  & $10^{6}$  & micro & $\mu$& 微    & $10^{-6}$\\
kilo  & k    & 千    & $10^{3}$  & milli & m    & 毫    & $10^{-3}$\\
\hline
\end{tabular}
\caption{單位前綴表}
\end{table}

\subsection{常用單位}
\subsubsection{資料儲存的單位}
\begin{itemize}
\item \textbf{位元} (b, bit)::只能儲存一個 0 或 1,是電腦最小的儲存單位。
\item \textbf{位元組} (B, byte):1 位元組 = 8 位元,\textbf{資料處理及定址最小單位}。
\item 字組 (W, word):1 字組 = 2 Bytes,可以用來表示\textbf{中文字元}。
\item 單位換算:電腦以 2 進位來表示,因此前綴就取最靠近 1000 倍,但是同為 2 的冪次的數字,因此取 $ 2^{10} = 1024$ 倍。
  \begin{enumerate}
  \item 1 kilobytes \textbf{(1 KB)} = $2^{10}$ Bytes $\approx$ $10^{3}$ Bytes。
  \item 1 megabytes \textbf{(1 MB)} = $2^{20}$ Bytes $\approx$ $10^{6}$ Bytes。
  \item 1 gigabytes \textbf{(1 GB)} = $2^{30}$ Bytes $\approx$ $10^{9}$ Bytes。
  \item 1 terabytes \textbf{(1 TB)} = $2^{40}$ Bytes $\approx$ $10^{12}$ Bytes。
  \end{enumerate}

\begin{figure}[h!]
  \centering
  \label{fig:hobbyte}
  \includegraphics[scale=0.5]{img/hobbit_hobbyte.jpg}
  \caption{位元和位元組 (大誤)}
\end{figure}

\item 某些場合中,1 KB 被視為 1000 Bytes,原因是要符合國際前綴,因此我們把 1024 Bytes 另外標示為 \textbf{1 KiB},其他以此類推。舉例:
  \begin{itemize}
  \item 1 TB 的硬碟,往往代表 1000 GB,因此照著這個邏輯,我們可以知道這個硬碟的真實容量是
    \begin{align*}
      {1\text{ TB}}\times{1000\text{ GB}\over\text{TB}}\times{1000\text{ MB}\over\text{GB}}\times{1000\text{ KB}\over\text{MB}}\times{1000\text{ Bytes}\over\text{KB}}&=10^{12}\text{ Bytes}\\
      {10^{12}\text{ Bytes}}\times{1\text{ KiB}\over{1024\text{ Bytes}}}\times{1\text{ MiB}\over{1024\text{ KiB}}}\times{1\text{ GiB}\over{1024\text{ MiB}}}&\approx{}931.3\text{ GiB}
    \end{align*}
  \end{itemize}
\end{itemize}

\subsubsection{資料傳輸速率}
\begin{itemize}
\item bps (bits per second):亦即每秒傳輸多少\textbf{位元}的資料。
\item 1 Kbps = $2^{10}$ bps $\approx$ $10^{3}$ bps。
\item 1 Mbps = $2^{20}$ bps $\approx$ $10^{6}$ bps。
\item 1 Gbps = $2^{30}$ bps $\approx$ $10^{9}$ bps。
\end{itemize}

\subsubsection{列印速度單位}
\begin{itemize}
\item CPM (character per minute):每分鐘可列印字數。
\item LPM (line per minute):每分鐘可列印行數。
\item PPM (page per minute):每分鐘可列印頁數。
\end{itemize}

\subsubsection{列印密度/解析度單位}
\begin{itemize}
\item \textbf{pixel = 像素 = 畫素 = 圖素} = picture element (圖像元素)
  \begin{enumerate}
  \item 圖像顯示的基本單位
  \item 有 RGB、CMYK 兩種
  \end{enumerate}
\item DPI (dot per inch):每英吋列印多少點數,\textbf{列印設備解析度}。
\item PPI (pixel per inch):每英吋有多少像素,\textbf{數位顯示解析度}。
\end{itemize}

\subsection{資料處理}

\begin{itemize}
\item \textbf{資料} (Data):未經處理之原始數據、文字等。是客觀的事實,沒有任何判斷或前後關聯。
\item \textbf{資訊} (Information):將原始資料經過一系列有計畫的處理後所得到的有用訊息。
\item \textbf{資料處理} (DP, Data Processing):將資料轉換為資訊的過程。依處理方式可分為兩類:
  \begin{enumerate}
  \item 人工資料處理
  \item 電子資料處理 (EDP, Electronic Data Processing)
  \end{enumerate}
\item 資料處理的方式:
  \begin{enumerate}
  \item \textbf{線上處理} (On-line Processing):將資料送去伺服器處理,最後再傳回結果。
    \begin{itemize}
    \item 即時處理 (Real-time Processing):接到需求立即處理。例:ATM、衛星導航。
    \item 分時處理 (Time-sharing Processing):多台終端連線至伺服器,其 CPU 輪流處理每位使用者的指令。
    \item 分散式資料處理 (DDP, Distribution Data Processing):將資料分散在許多電腦處理,可提升整體效率。
    \end{itemize}
  \item \textbf{離線處理} (Off-line Processing):不需要經由連線,在本地端進行相關的資料處理動作。
  \item \textbf{批次處理} (Batch Processing):將性質相同的工作累積起來做一次處理。
  \end{enumerate}
\end{itemize}

\section{計算機發展史}

\subsection{重要的里程碑}

\paragraph{}這部分有點印象就好了,除了強調部分尚屬重要常識需要注意之外,其他不用強記。

\subsubsection{早期計算工具}
\begin{itemize}
\item 西元前 2700 年至 3200 年左右,巴比倫已有使用\textbf{算盤}的紀錄。
\item 西元元年前後,希臘人發明\textbf{安提基特拉機械} (Antikythera mechanism)。
\item 西元 1623 年,施卡德 (Schickard) 發明史上第一個\textbf{機械式計算器}。
\item 西元 1633 年,威廉‧奧特雷德(William Oughtred)發明圓算尺,是日後計算尺的前身。
\item 西元 1642 年,布萊茲‧帕斯卡 (Blaise Pascal) 發明了\textbf{滾輪式加法器} (Pascaline)。
\end{itemize}

\subsubsection{打孔卡片 (1801 - 1970 年代)}
\begin{itemize}
\item 西元 1801 年,法國織布工人約瑟夫·瑪麗·雅卡爾 (Joseph Marie Jacquard) 發明能用不同\textbf{打孔卡片}控制編織圖案的織布機。
\item 西元 1822 年,英國數學家\textbf{查爾斯·巴貝奇} (Charles Babbage) 提出\textbf{差分機}的構想,但未真正完成 (於 1842 年宣告中止)。
\item 西元 1837 年,巴貝奇轉而建造用途更廣的分析機亦失敗,但分析機已經有現代電腦的基本結構。
\item 西元 1890 年,美國科學家何樂利 (Hollerith) 以打孔卡片處理人口普查,在兩年半內完成這項工作。何樂利在 1896 年成立 Tabulating Machine Company,是後來國際商業機器公司 (IBM) 的前身。
\end{itemize}

\subsubsection{真空管時期 (1942 - 1946)}
\begin{itemize}
\item 西元 1942 年,美國愛荷華州立大學的阿塔納索夫 (Atanasoff) 和貝理 (Berry) 製造出第一部\textbf{電子式電腦 ABC} (Atanasoff-Berry Computer),使用\textbf{真空管}作為主要電子元件。
\item 西元 1944 年,美國哈佛大學教授艾肯 (Aiken) 建造第一部\textbf{全自動電子式電腦} Mark I,採用\textbf{哈佛架構}。
\item 西元 1945 年,馮紐曼 (范紐曼,von Neumann) 提出\textbf{內儲程式} (stored program) 的觀念,程式指令和資料一起儲存在記憶體中 (早期是固定在硬體架構中)。
\item 西元 1946 年,美國賓州大學教授 John W. Mauchly 和 J. Presper Eckert Jr. 與軍方合作製造出第一部大型電子計算機──\textbf{ENIAC},用以計算飛彈彈道。後來再以\textbf{馮紐曼架構}於 1949 年改良為 EDVAC。
\end{itemize}

\subsubsection{電晶體時期 (1947 - 1957)}
\begin{itemize}
\item 西元 1947 年,美國貝爾實驗室發明了\textbf{電晶體}。
\item 西元 1949 年,英國劍橋大學完成第一部大型內儲程式電腦──EDSAC。
\item 西元 1951 年,美國賓州大學教授 John W. Mauchly 和 J. Presper Eckert Jr. 建造第一部商業用途電腦──\textbf{UNIVAC 1}。同年,UNIVAC 1 投入美國人口普查局服役。
\item 西元 1954 年,美國貝爾實驗室製出第一台以電晶體為主要元件的電腦──\textbf{TRADIC}。
\end{itemize}

\subsubsection{積體電路時期 (1958 - 1976)}
\begin{itemize}
\item 西元 1958 年,美國德州儀器公司發明了\textbf{積體電路} (IC, Integrated Circuit)。
\item 西元 1964 年,美國 IBM 開發出第一台以 IC 為主元件的 \textbf{System / 360 電腦}。
\item 西元 1965 年,美國摩爾 (Moore) 提出\textbf{摩爾定律}:每隔 18 到 24 個月積體電路中電晶體的數目會變成兩倍。摩爾在 1968 年成立 Intel 公司。
\item 西元 1971 年,美國 Intel 公司發表全世界第一款微處理器 4004。同一年,\textbf{丹尼斯·里奇} (Dennis Ritchie) 與肯·湯普遜 (Ken Thompson) 發明了 \textbf{C 語言}。
\item 西元 1973 年,第三版 \textbf{UNIX} 誕生,此版本是丹尼斯·里奇與肯·湯普遜以 C 語言寫成。
\end{itemize}

\subsubsection{個人化電腦 (1977 - 1994)}
\begin{itemize}
\item 西元 1977 年,美國 Apple 公司推出 \textbf{APPLE II} 電腦:採封閉性硬體系統架構,從此進入電腦個人化時代。
\item 西元 1981 年,美國 IBM 公司推出第一台個人電腦,稱為 \textbf{IBM PC}:採開放性硬體系統架構,造成 IBM 相容型電腦的大量生產。
\item 西元 1983 年,\textbf{TCP/IP 協定}成為 \textbf{ARPANET} 的主要協定,為日後的網際網路打下基礎。
\item 西元 1985 年,\textbf{Intel 80386 CPU} 發布,讓往後電腦進入多工作業系統時代。
\item 西元 1990 年,\textbf{windows 3.0} 推出,微軟開始進入圖形化介面。
\item 西元 1991 年,\textbf{林納斯·本納第·托瓦茲} (Linus Benedict Torvalds) 首次發行 \textbf{linux} 作業系統。
\end{itemize}

\subsubsection{網路資訊時代 (1995 - 至今)}
\begin{itemize}
\item 西元 1995 年,\textbf{網際網路} (Internet) 開放給商業,開啟網路時代。
\item 西元 1999 年,「\textbf{Web 2.0}」首次被提出。
\item 西元 2001 年,美國 Apple 公司,\textbf{iPod} 正式發表,開啟數位音樂及掌上型裝置的新時代。
\item 西元 2006 年,亞馬遜開始提供雲端運算服務。同年,Google 首次提出「\textbf{雲端計算}」的概念。
\item 西元 2007 年,美國 Apple 公司推出 \textbf{iPhone},智慧型手機開始風行。同年,開放手持裝置聯盟發布 \textbf{Android} 作業系統。
\end{itemize}

\subsection{計算機世代的演進}

\begin{table}[h!]
\centering
\caption{傳統的計算機世代}
\label{tbl:com-gen}
\begin{tabular}{|l||c|c|c|c|c|}
\hline
世代 &
\begin{tabular}[c]{@{}c@{}}第一代\\ 1942 - 1954\end{tabular} &
\begin{tabular}[c]{@{}c@{}}第二代\\ 1954 - 1964\end{tabular} &
\begin{tabular}[c]{@{}c@{}}第三代\\ 1964 - 1971\end{tabular} &
\begin{tabular}[c]{@{}c@{}}第四代\\ 1971 - 現在\end{tabular}  &
\begin{tabular}[c]{@{}c@{}}第五代\\ 發展中\end{tabular}\\
\hline
\hline
主要元件 & 真空管 & 電晶體 &
\begin{tabular}[c]{@{}c@{}}積體電路\\ (IC)\end{tabular} &
\begin{tabular}[c]{@{}c@{}}超大型\\ 積體電路\\ (VLSI)\end{tabular} &
\begin{tabular}[c]{@{}c@{}}人工智慧\\ (AI)\end{tabular}\\
\hline
程式語言  &
\begin{tabular}[c]{@{}c@{}}機器語言、\\ 組合語言\end{tabular} &
\begin{tabular}[c]{@{}c@{}}組合語言、\\ COBOL、\\ FORTRAN\end{tabular} &
\begin{tabular}[c]{@{}c@{}}高階語言、\\ 物件導向語言\\ 雛形\end{tabular} &
\begin{tabular}[c]{@{}c@{}}物件導向語\\ 言、問題導\\ 向語言\end{tabular} &
\begin{tabular}[c]{@{}c@{}}自然語言、\\ LISP、\\ PROLOG\end{tabular} \\
\hline
主記憶體  &
\begin{tabular}[c]{@{}c@{}}磁鼓\\ (magnetic\\ drum)\end{tabular} &
\begin{tabular}[c]{@{}c@{}}4KB - 32KB\\ 磁鼓或磁芯\\ (magnetic core)\end{tabular} &
\begin{tabular}[c]{@{}c@{}}32KB - 3MB\\ 磁芯或半導體\\ 記憶體\end{tabular} &
\begin{tabular}[c]{@{}c@{}}3 MB 以上\\ 的半導體\\ 記憶體\end{tabular} & \\
\hline
輔助記憶體 & 打孔卡片 & 磁帶、磁碟 & 磁帶、磁碟 & 磁碟、光碟 & \\
\hline
代表電腦 &
\begin{tabular}[c]{@{}c@{}}ENIAC、\\ UNIVAC\end{tabular} &
TRADIC &
\begin{tabular}[c]{@{}c@{}}IBM\\ System/360\end{tabular} &
各種 PC & \\
\hline
重要記事 & &
\begin{tabular}[c]{@{}c@{}}開始有\\ 高階語言\end{tabular} &
\begin{tabular}[c]{@{}c@{}}開始有\\ 作業系統\end{tabular} &
微電腦誕生 & \\
\hline
\end{tabular}
\end{table}

\section{計算機的類型}

\subsection{傳統分類}
\begin{itemize}
\item \textbf{巨}型計算機--超級電腦 (supercomputer):
\item
\item
\end{itemize}

\subsection{實際分類}

\section{計算機的組成}

\subsection{計算機組成要素}

\subsection{計算機架構}

\subsubsection{馮紐曼架構}

\subsubsection{哈佛架構}

\clearpage
\end{CJK}

\end{document}